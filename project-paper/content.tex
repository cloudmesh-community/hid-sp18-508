% status: 0
% chapter: TBD

\title{Performance Comparison}

\author{Yue Guo}
\affiliation{%
  \institution{Indiana University}
  \streetaddress{Smith Research Center}
  \city{Bloomington} 
  \state{IN} 
  \postcode{47408}
  \country{USA}}
\email{yueguo@iu.edu}

\begin{abstract}
This project is to compare the performance of Cassandra and MongoDB 
under different coditions and different operations. In this project, Cassandra
 and MongoDB are deployed in three different platform, Macintosh macOS 
 High Sierra, Google Cloud Ubuntu 14.04 and Raspberry pi Model B Ubuntu 
 16.04.
\end{abstract}

\keywords{Cassandra, MongoDB, Raspberry Pi, Google Cloud, hid-sp18-508}


\maketitle


\section{Introduction}
Cassandra and MongoDB are both not Relational DataBase Management System. 
There are four kinds of database, including Key-Value Stores, Big Tablle 
Clones, Document Database and Graph Database. Cassandra is a kind of Big 
Table Clones, while MongoDB is a kind of Document Database. Cassandra more 
focuses on Availability while MongoDB focuses more on Consistency. So there
 should be some difference in their performance when dealing with same data 
 and same operations on the same platform. After deploying both on the same 
 platform and testing same data on them, their performance can be compared. 
 ``Understanding the performance behavior of a NoSQL database like Apache 
 Cassandra? under various conditions is critical. Conducting a formal proof 
 of concept (POC) in the environment in which the database will run is the 
 best way to evaluate platforms~\cite{hid-sp18-508-benchmarking}.''

\section{Technology Used}
This section describe the technologies that has been used through out 
the project.

\subsection{Cassandra}
Apache Cassandra is a kind of distributed NoSQL database management 
system, which is free and open source. It can handle a large amount of 
data on a large number of servers, which can provide high quality and no 
single point of failure~\cite{hid-sp18-508-cassandra}. 

\subsection{MongoDB}
MongoDB is also a king of NoSQL database management system, which 
is free and open source. Besides, it is a document-oriented database 
program. ``MongoDB uses JSON-like documents with 
schemas''~\cite{hid-sp18-508-mongodb}.

\subsection{Raspberry Pi}
Raspberry Pi can be considered as a small computer that can be used 
as a game machine or a platform to learn programming or a 
server~\cite{hid-sp18-508-raspberryPi}.

The hardware used for Raspberry Pi consists of: 

one Raspberry Pi 3 Model B computer
The pi is shown in following Figure~\ref{f:fly}.

\begin{figure}[!ht]
  \centering\includegraphics[width=\columnwidth]{images/pi.jpeg}
  \caption{Raspberry Pi 3 Model B}\label{f:fly}
\end{figure}


one AmazonBasics High-speed HDMI cable

one Elecrow RPA050 HDMI 5-inch 800*480 TFT LCD Display with Touch
 Screen Monitor for Raspberry Pi
 The monitor is shown in following Figure~\ref{f:fly}.

\begin{figure}[!ht]
  \centering\includegraphics[width=\columnwidth]{images/display.jpeg}
  \caption{Raspberry Pi 3 Model B}\label{f:fly}
\end{figure}

one CanaKit 5V 2.5A Raspberry Pi 3 Model B Power Supply

one Happy Hacking keyboard

\subsection{Google Cloud}
Google Cloud provides a series of modular cloud services alongside a lot of
 management tools~\cite{hid-sp18-508-googleCloud}. Although
  it is not open source and free, when open an account,
 it provides three hundred dollar which can be used in the first year.
 
 \section{Deployment}
 More details are on the github.
\subsection{Deploy Cassandra}

Download Cassandra archive package from website

Decompress and install Cassandra 

Switch to Cassandra bin directory and start up Cassandra

\subsection{Deploy MongoDB }

Import the public key used by the package mangement  system

Create a list file for MongoDB

Reload local package database

Install the MongoDB packages

\subsection{Deploy Raspberry Pi}

Setup an Ubuntu SSO account

Import an SSH key into Ubuntu SSO account

Create installation medias for Ubuntu Core 

burn Ubuntu into SD card

\section{Design}
Choose MongoDB as backend first, then use Cassandra as backend. 
The performance metrics include: 

1. How fast can it process when using different backends

2. Endurance testing. Running search operation continuously to confirm its 
sustainability.

\section{Scenarios}

Using server connected to MongoDB:

1. Insert 10000 data 1000 times

2. Seach 10000 data 1000 times

3. Insert 5000 data 1000 times

4. Search 5000 data 1000 times

5. Insert 1000 data 1000 times

6. Search 1000 data 1000 times

7. Two thread insert 10000 data and search 10000 data at the same time

8. Two thread insert 5000 data and search 5000 data at the same time

9. Two thread insert 1000 data and search 1000 data at the same time

Using server connected to Cassandra:

1. Insert 10000 data 1000 times

2. Search 10000 data 1000 times

3. Insert 5000 data 1000 times

4. Search 5000 data 1000 times

5. Insert 1000 data 1000 times

6. Search 1000 data 1000 times

7. Two thread insert 10000 data and search 10000 data at the same time

8. Two thread insert 5000 data and search 5000 data at the same time

9. Two thread insert 1000 data and search 1000 data at the same time

\section{Assessment}
What parameter:

use time package, time\_after - time\_pre 

How to process:

Redirect the output of time result into the specified file. Use Matlab program 
and compare the output of MongoDB and Cassandra to draw some plots. 

\section{Results}
Compare insert 1000 data 1000 times on Macintosh~\ref{f:fly}.

\begin{figure}[!ht]
  \centering\includegraphics[width=\columnwidth]{images/insert_comp_1000.jpg}
  \caption{Insert 1000 data 1000 times on Macintosh }\label{f:fly}
\end{figure}

Compare insert 5000 data 1000 times on Macintos~\ref{f:fly}.

\begin{figure}[!ht]
  \centering\includegraphics[width=\columnwidth]
  {images/insert_comp_5000.jpg}
  \caption{Insert 5000 data 1000 times on Macintosh}\label{f:fly}
\end{figure}

Compare insert 10000 data 1000 times on Macintos~\ref{f:fly}.

\begin{figure}[!ht]
  \centering\includegraphics[width=\columnwidth]
  {images/insert_comp_10000.jpg}
  \caption{Insert 10000 data 1000 times on Macintosh}\label{f:fly}
\end{figure}

Compare serach 1000 data 1000 times on Macintosh~\ref{f:fly}.

\begin{figure}[!ht]
  \centering\includegraphics[width=\columnwidth]
  {images/search_comp_1000.jpg}
  \caption{Serach 1000 data 1000 times on Macintosh }\label{f:fly}
\end{figure}

Compare serach 5000 data 1000 times on Macintos~\ref{f:fly}.

\begin{figure}[!ht]
  \centering\includegraphics[width=\columnwidth]
  {images/search_comp_5000.jpg}
  \caption{Serach 5000 data 1000 times on Macintosh}\label{f:fly}
\end{figure}

Compare serach 10000 data 1000 times on Macintos~\ref{f:fly}.

\begin{figure}[!ht]
  \centering\includegraphics[width=\columnwidth]
  {images/search_comp_10000.jpg}
  \caption{Serach 10000 data 1000 times on Macintosh}\label{f:fly}
\end{figure}



Compare insert 1000 data 1000 times on Raspberry Pi~\ref{f:fly}.

\begin{figure}[!ht]
  \centering\includegraphics[width=\columnwidth]
  {images/insert_comp_1000_pi.jpg}
  \caption{Insert 1000 data 1000 times on Raspberry Pi }\label{f:fly}
\end{figure}

Compare insert 5000 data 1000 times on Raspberry Pi~\ref{f:fly}.

\begin{figure}[!ht]
  \centering\includegraphics[width=\columnwidth]
  {images/insert_comp_5000_pi.jpg}
  \caption{Insert 5000 data 1000 times on Raspberry Pi}\label{f:fly}
\end{figure}

Compare insert 10000 data 1000 times on Raspberry Pi~\ref{f:fly}.

\begin{figure}[!ht]
  \centering\includegraphics[width=\columnwidth]
  {images/insert_comp_10000_pi.jpg}
  \caption{Insert 10000 data 1000 times on Raspberry Pi}\label{f:fly}
\end{figure}

Compare serach 1000 data 1000 times on Raspberry Pi~\ref{f:fly}.

\begin{figure}[!ht]
  \centering\includegraphics[width=\columnwidth]
  {images/search_comp_1000_pi.jpg}
  \caption{Serach 1000 data 1000 times on Raspberry Pi }\label{f:fly}
\end{figure}

Compare serach 5000 data 1000 times on Raspberry Pi~\ref{f:fly}.

\begin{figure}[!ht]
  \centering\includegraphics[width=\columnwidth]
  {images/search_comp_5000_pi.jpg}
  \caption{Serach 5000 data 1000 times on Raspberry Pi}\label{f:fly}
\end{figure}

Compare serach 10000 data 1000 times on Raspberry Pi~\ref{f:fly}.

\begin{figure}[!ht]
  \centering\includegraphics[width=\columnwidth]
  {images/search_comp_10000_pi.jpg}
  \caption{Serach 10000 data 1000 times on Raspberry Pi}\label{f:fly}
\end{figure}

Compare insert 1000 data 1000 times on Google Cloud~\ref{f:fly}.

\begin{figure}[!ht]
  \centering\includegraphics[width=\columnwidth]
  {images/insert_comp_1000_google.jpg}
  \caption{Insert 1000 data 1000 times on Google Cloud}\label{f:fly}
\end{figure}

Compare insert 5000 data 1000 times on Google Cloud~\ref{f:fly}.

\begin{figure}[!ht]
  \centering\includegraphics[width=\columnwidth]
  {images/insert_comp_5000_google.jpg}
  \caption{Insert 5000 data 1000 times on Google Cloud}\label{f:fly}
\end{figure}

Compare search 1000 data 1000 times on Google Cloud~\ref{f:fly}.

\begin{figure}[!ht]
  \centering\includegraphics[width=\columnwidth]
  {images/search_comp_1000_google.jpg}
  \caption{Search 1000 data 1000 times on Google Cloud}\label{f:fly}
\end{figure}

Compare search 5000 data 1000 times on Google Cloud~\ref{f:fly}.

\begin{figure}[!ht]
  \centering\includegraphics[width=\columnwidth]
  {images/search_comp_5000_google.jpg}
  \caption{Search 5000 data 1000 times on Google Cloud}\label{f:fly}
\end{figure}


\section{Conclusion}
Compare the performance of MongoDB and Cassandra. 

Cassandra takes more time to insert and search, especially search. 

However, on Raspberry Pi the difference of their performance become closer.

The performance of MongoDB and Cassandra on Google Cloud is more stable.


\begin{acks}

  The authors would like to thank Dr.~Gregor~von~Laszewski for his
  support and suggestions to write this paper.

\end{acks}

\bibliographystyle{ACM-Reference-Format}
\bibliography{report} 

