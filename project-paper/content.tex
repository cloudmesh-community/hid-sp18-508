% status: 0
% chapter: Database

\title{Casandra and MongoDB Performance Comparison}

\author{Yue Guo}
\affiliation{%
  \institution{Indiana University}
  \streetaddress{Smith Research Center}
  \city{Bloomington} 
  \state{IN} 
  \postcode{47408}
  \country{USA}}
\email{yueguo@iu.edu}

\begin{abstract}
This project is to compare the performance of Cassandra and MongoDB 
under different coditions and different operations. In this project, Cassandra
 and MongoDB are deployed in three different platform, Macintosh macOS 
 High Sierra, Google Cloud Ubuntu 14.04 and Raspberry pi Model B Ubuntu 
 16.04.
\end{abstract}

\keywords{Cassandra, MongoDB, Raspberry Pi, Google Cloud, hid-sp18-508}


\maketitle


\section{Introduction}
Cassandra and MongoDB are both not Relational DataBase Management System. 
There are four kinds of database, including Key-Value Stores, Big Tablle 
Clones, Document Database and Graph Database. Cassandra is a kind of Big 
Table Clones, while MongoDB is a kind of Document Database. Cassandra more 
focuses on Availability while MongoDB focuses more on Consistency. So there
 should be some difference in their performance when dealing with same data 
 and same operations on the same platform. After deploying both on the same 
 platform and testing same data on them, their performance can be compared. 
 ``Understanding the performance behavior of a NoSQL database like Apache 
 Cassandra? under various conditions is critical. Conducting a formal proof 
 of concept (POC) in the environment in which the database will run is the 
 best way to evaluate platforms~\cite{hid-sp18-508-benchmarking}.''

\section{Technology Used - Software}
This section describe the technologies that has been used through out 
the project.
We used two NoSQL database here for Cloud Computing Courses, 
because for clouding computing NoSQL database works much better
than traditional relational databases. Those NoSQL database can ``handle load
by spreading data among many servers, making them a natural fit for the 
cloud computing environment''~\cite{hid-sp18-508-nosql}.


\subsection{Cassandra}
Apache Cassandra is a kind of distributed NoSQL database management 
system, which is free and open source. It can handle a large amount of 
data on a large number of servers, which can provide high quality and no 
single point of failure~\cite{hid-sp18-508-cassandra}. 

And Cassandra has several advantages:

``1. Provides simple solutions for complex problems

2. Easy to learn

3. Reduces administration costs and overhead for non-core tasks

4. Fault tolerant to the extreme

5. Fast reads and extremely fast writes''~\cite{hid-sp18-508-cassandraAd}. 


\subsection{MongoDB}
MongoDB is also a king of NoSQL database management system, which 
is free and open source. Besides, it is a document-oriented database 
program. ``MongoDB uses JSON-like documents with 
schemas''~\cite{hid-sp18-508-mongodb}.

\section{Technology Used - Hardware}
This section describe the technologies that has been used through out 
the project.

\subsection{Macintosh HD}
The System Configuration:

MacBook Pro (Retina, 15-inch, Mid 2015)

Processor: 2.5GHz Intel Core i7

Memory: 16 GB 1600MHz DDR3


\subsection{Raspberry Pi}
Raspberry Pi can be considered as a small computer that can be used 
as a game machine or a platform to learn programming or a 
server~\cite{hid-sp18-508-raspberryPi}. Since it can be considered as a
computer just like someone's own personal computer, it can be considered 
exactly the same as PC. Furthermore, using Raspberry Pi as a server can
have  several advantages. It can ``provide you with a low-cost, silent, non-heating
 machine that fits easily into a room, and with very low power consumption of
 electricity''~\cite{hid-sp18-508-raspberryServer}.

The hardware used for Raspberry Pi consists of: 

one Raspberry Pi 3 Model B computer
The pi is shown in following Figure~\ref{f:fly}.

\begin{figure}[!ht]
  \centering\includegraphics[width=\columnwidth]{images/pi.jpeg}
  \caption{Raspberry Pi 3 Model B}\label{f:fly}
\end{figure}


one AmazonBasics High-speed HDMI cable

one Elecrow RPA050 HDMI 5-inch 800*480 TFT LCD Display with Touch
 Screen Monitor for Raspberry Pi
 The monitor is shown in following Figure~\ref{f:fly}.

\begin{figure}[!ht]
  \centering\includegraphics[width=\columnwidth]{images/display.jpeg}
  \caption{Raspberry Pi 3 Model B}\label{f:fly}
\end{figure}

one CanaKit 5V 2.5A Raspberry Pi 3 Model B Power Supply

one Happy Hacking keyboard

\subsection{Google Cloud}
Google Cloud provides a series of modular cloud services alongside a lot of
 management tools~\cite{hid-sp18-508-googleCloud}. Just as Raspberry Pi,
 we can consider it as a PC. The thing that is different from Raspberry Pi is
 that it is a distributed system of computing. Although
  it is not open source and free, when open an account,
 it provides three hundred dollar which can be used in the first year.
 
 The System Configuration of VM instance on Google Cloud:
 
 Machine type: n1-standard-8 (8 vCPUs, 30 GB memory)
 
 Operating System: Ubuntu 14.04

 \section{Deployment}
The following subsection will show how to deploy MongoDB and Cassandra
on different operating system.

\subsection{Deploy Cassandra on Mac OS X}

Some steps are retrieved from https://www.datastax.com/2012/01/working-
with-apache-cassandra-on-mac-os-x/

Download Cassandra

curl -OL http://downloads.datastax.com/community/dsc.tar.gz

Install Cassandra

tar -xzf dsc.tar.gz

Then switch to the new Cassandra bin directory and start up Cassandra:

cd dsc-cassandra-1.2.2/bin

sudo ./cassandra

Now that you have Cassandra running, the next thing to do is connect to the 
server and begin creating database objects. This is done with the Cassandra 
Query Language (CQL) utility. CQL is a very SQL-like language that lets you
 create objects as you?re likely used to doing in the RDBMS world.

./cqlsh

Connected to Test Cluster at localhost:9160.

For this brief introduction, we will just create a basic keyspace to hold some 
example data objects we will create:

cqlsh$>$ create keyspace dev

Let us create a base table to hold train data:

cqlsh$> $use dev;

cqlsh:dev$>$ create table test(uid varchar primary key); 

cqlsh:dev$>$ insert into test(uid) values('1');

\subsection{Deploy Cassandra on Ubuntu 14.04}

Some steps are retrieved from https://www.digitalocean.com/community/
tutorials/how-to-install-cassandra-and-run-a-single-node-cluster-on-ubuntu-14-04

Installing Cassandra

echo deb http://www.apache.org/dist/cassandra/debian 22x main | sudo tee -a 
/etc/apt/sources.list.d/cassandra.sources.list

Add the public key using this pair of commands, which must be run one after the other

gpg --keyserver pgp.mit.edu --recv-keys F758CE318D77295D

gpg --export --armor F758CE318D77295D | sudo apt-key add -

Update the package database once again:

	sudo apt-get update

Finally, install Cassandra:

	sudo apt-get install cassandra

Starting Cassandra:

sudo service cassandra status

Connecting to the Cluster:

sudo nodetool status

cqlsh

Then the following steps are the same with Mac Os

\subsection{Deploy MongoDB on Mac OS X}
Steps are retrieved from https://docs.mongodb.com/manual/tutorial
/install-mongodb-on-os-x/

Install MongoDB Community Edition Manually

Download the binary files for the desired release of MongoDB.


Extract the files from the downloaded archive.

For example, from a system shell, you can extract through the tar command:

$tar -zxvf mongodb-osx-ssl-x86_64-3.6.4.tgz$


Copy the extracted archive to the target directory.

Copy the extracted folder to the location from which MongoDB will run.

mkdir -p mongodb

$cp -R -n mongodb-osx-ssl-x86_64-3.6.4/ mongodb$

Ensure the location of the binaries is in the PATH variable.

The MongoDB binaries are in the bin/ directory of the archive. To ensure that the binaries are in your PATH, you can modify your PATH.

For example, you can add the following line to your shell?s rc file (e.g. ~/.bashrc):

$export PATH=<mongodb-install-directory>/bin:PATH$
Replace $<mongodb-install-directory>$ with the path to the extracted MongoDB archive.


\subsection{Deploy MongoDB on Ubuntu 14.04}
Some steps are retrieved from https://docs.mongodb.com/manual/tutorial
/install-mongodb-on-ubuntu/

Install MongoDB Community on Ubuntu
Import the public key used by the package management system

sudo apt-key adv --keyserver hkp://keyserver.ubuntu.com:80 --recv 2930ADAE8CAF5059EE73BB4B58712A2291FA4AD5
Create a list file for MongoDB

echo deb [ arch=amd64 ] https://repo.mongodb.org/apt/ubuntu trusty
/mongodb-org/3.6 multiverse | sudo tee /etc/apt/sources.list.d
/mongodb-org-3.6.list

Reload local package database

sudo apt-get update

Install the MongoDB packages, Install the latest stable version of MongoDB.

sudo apt-get install -y mongodb-org

Run MongoDB Community Edition

sudo service mongod start

Stop MongoDB

sudo service mongod stop

Restart MongoDB

sudo service mongod restart


\subsection{Deploy Raspberry Pi}

Some steps are copied from https://www.ubuntu.com
/download/iot/raspberry-pi-2-3


Setup an Ubuntu SSO account

1. An Ubuntu SSO account is required to create the first user on an Ubuntu Core
 installation.

2. Start by creating an Ubuntu SSO account 

3. Import an SSH key into your Ubuntu SSO accout 

Create installation medias for Ubuntu Core on Mac OS, some steps are copied from
https://developer.ubuntu.com/core/get-started/installation-medias

Download Raspberry Pi 3

Insert your SD card or USB flash drive

Open a terminal window (Go to Application -> Utilities, you will find the Terminal 
app there), then run the following command:

 diskutil list
 
 
Unmount your SD card with the following command:

diskutil unmountDisk <drive address>

You can now copy the image to the SD card, using the following command:

sudo sh -c $'xzcat ~/Downloads/<image file> | sudo dd of=<drive address> bs=32m'$

\section{Benchmark}
Choose MongoDB as backend first, then use Cassandra as backend. 
Choose Macintosh first, then do same job on Raspberry Pi and finally on Google Cloud.
The performance metrics include: 

1. How fast can it process when using different backends

2. How fast can it process when under different hardware and different system

\subsection{Benchmark - Scenarios}

All the data used for inserting and searching is a single varchar type user-id.

All the details of MongoDB,  Cassandra, Macintosh, Raspberry Pi  and 
Google Cloud used here is introduced in section Technology Used - 
Software and Technology Used - Hardware.

Scenario 1. Insert 1000 user-id 1000 times on Macintosh with MongoDB 

Scenario 2. Insert 5000 user-id 1000 times on Macintosh with MongoDB 

Scenario 3. Insert 10000 user-id 1000 times on Macintosh with MongoDB 

Scenario 4. Search 1000 user-id 1000 times on Macintosh with MongoDB 

Scenario 5. Search 5000 user-id 1000 times on Macintosh with MongoDB 

Scenario 6. Search 10000 user-id 1000 times on Macintosh with MongoDB 


Scenario 7. Insert 1000 user-id 1000 times on Raspberry Pi with MongoDB 

Scenario 8. Insert 5000 user-id 1000 times on Raspberry Pi with MongoDB 

Scenario 9. Insert 10000 user-id 1000 times on Raspberry Pi with MongoDB 

Scenario 10. Search 1000 user-id 1000 times on Raspberry Pi with MongoDB 

Scenario 11. Search 5000 user-id 1000 times on Raspberry Pi with MongoDB 

Scenario 12. Search 10000 user-id 1000 times on Raspberry Pi with MongoDB 


Scenario 13. Insert 1000 user-id 1000 times on Google Cloud with MongoDB 

Scenario 14. Insert 5000 user-id 1000 times on Google Cloud with MongoDB 

Scenario 15. Insert 10000 user-id 1000 times on Google Cloud with MongoDB 

Scenario 16. Search 1000 user-id 1000 times on Google Cloud with MongoDB 

Scenario 17. Search 5000 user-id 1000 times on Google Cloud with MongoDB 

Scenario 18. Search 10000 user-id 1000 times on Google Cloud with MongoDB 




Scenario 19. Insert 1000 user-id 1000 times on Macintosh with Cassandra 

Scenario 20. Insert 5000 user-id 1000 times on Macintosh with Cassandra 

Scenario 21. Insert 10000 user-id 1000 times on Macintosh with Cassandra 

Scenario 22. Search 1000 user-id 1000 times on Macintosh with Cassandra 

Scenario 23. Search 5000 user-id 1000 times on Macintosh with Cassandra 

Scenario 24. Search 10000 user-id 1000 times on Macintosh with Cassandra 


Scenario 25. Insert 1000 user-id 1000 times on Raspberry Pi with Cassandra 

Scenario 26. Insert 5000 user-id 1000 times on Raspberry Pi with Cassandra 

Scenario 27. Insert 10000 user-id 1000 times on Raspberry Pi with Cassandra 

Scenario 28. Search 1000 user-id 1000 times on Raspberry Pi with Cassandra 

Scenario 29. Search 5000 user-id 1000 times on Raspberry Pi with Cassandra 

Scenario 30. Search 10000 user-id 1000 times on Raspberry Pi with Cassandra 


Scenario 31. Insert 1000 user-id 1000 times on Google Cloud with Cassandra 

Scenario 32. Insert 5000 user-id 1000 times on Google Cloud with Cassandra 

Scenario 33. Insert 10000 user-id 1000 times on Google Cloud with Cassandra 

Scenario 34. Search 1000 user-id 1000 times on Google Cloud with Cassandra 

Scenario 35. Search 5000 user-id 1000 times on Google Cloud with Cassandra 

Scenario 36. Search 10000 user-id 1000 times on Google Cloud with Cassandra 

\section{Assessment}
What parameter:

use time package, time\_after - time\_pre 

How to process:

Redirect the output of time result into the specified file. Use Matlab program 
and compare the output of MongoDB and Cassandra to draw some plots. 

\section{Results}
Compare insert 1000 data 1000 times on Macintosh~\ref{f:fly}.

\begin{figure}[!ht]
  \centering\includegraphics[width=\columnwidth]{images/insert_comp_1000.jpg}
  \caption{Insert 1000 data 1000 times on Macintosh 
  (Scenario 1 compare with Scenario 19)}\label{f:fly}
\end{figure}

Compare insert 5000 data 1000 times on Macintos~\ref{f:fly}.

\begin{figure}[!ht]
  \centering\includegraphics[width=\columnwidth]
  {images/insert_comp_5000.jpg}
  \caption{Insert 5000 data 1000 times on Macintosh
    (Scenario 2 compare with Scenario 20)}\label{f:fly}
\end{figure}

Compare insert 10000 data 1000 times on Macintos~\ref{f:fly}.

\begin{figure}[!ht]
  \centering\includegraphics[width=\columnwidth]
  {images/insert_comp_10000.jpg}
  \caption{Insert 10000 data 1000 times on Macintosh
    (Scenario 3 compare with Scenario 21)}\label{f:fly}
\end{figure}

Compare search 1000 data 1000 times on Macintosh~\ref{f:fly}.

\begin{figure}[!ht]
  \centering\includegraphics[width=\columnwidth]
  {images/search_comp_1000.jpg}
  \caption{Serach 1000 data 1000 times on Macintosh
   (Scenario 4 compare with Scenario 22) }\label{f:fly}
\end{figure}

Compare search 5000 data 1000 times on Macintos~\ref{f:fly}.

\begin{figure}[!ht]
  \centering\includegraphics[width=\columnwidth]
  {images/search_comp_5000.jpg}
  \caption{Serach 5000 data 1000 times on Macintosh
   (Scenario 5 compare with Scenario 23)}\label{f:fly}
\end{figure}

Compare search 10000 data 1000 times on Macintos~\ref{f:fly}.

\begin{figure}[!ht]
  \centering\includegraphics[width=\columnwidth]
  {images/search_comp_10000.jpg}
  \caption{Serach 10000 data 1000 times on Macintosh
   (Scenario 6 compare with Scenario 24)}\label{f:fly}
\end{figure}



Compare insert 1000 data 1000 times on Raspberry Pi~\ref{f:fly}.

\begin{figure}[!ht]
  \centering\includegraphics[width=\columnwidth]
  {images/insert_comp_1000_pi.jpg}
  \caption{Insert 1000 data 1000 times on Raspberry Pi
   (Scenario 7 compare with Scenario 25) }\label{f:fly}
\end{figure}

Compare insert 5000 data 1000 times on Raspberry Pi~\ref{f:fly}.

\begin{figure}[!ht]
  \centering\includegraphics[width=\columnwidth]
  {images/insert_comp_5000_pi.jpg}
  \caption{Insert 5000 data 1000 times on Raspberry Pi
   (Scenario 8 compare with Scenario 26)}\label{f:fly}
\end{figure}

Compare insert 10000 data 1000 times on Raspberry Pi~\ref{f:fly}.

\begin{figure}[!ht]
  \centering\includegraphics[width=\columnwidth]
  {images/insert_comp_10000_pi.jpg}
  \caption{Insert 10000 data 1000 times on Raspberry Pi
   (Scenario 9 compare with Scenario 27)}\label{f:fly}
\end{figure}

Compare search 1000 data 1000 times on Raspberry Pi~\ref{f:fly}.

\begin{figure}[!ht]
  \centering\includegraphics[width=\columnwidth]
  {images/search_comp_1000_pi.jpg}
  \caption{Serach 1000 data 1000 times on Raspberry Pi
   (Scenario 10 compare with Scenario 28) }\label{f:fly}
\end{figure}

Compare search 5000 data 1000 times on Raspberry Pi~\ref{f:fly}.

\begin{figure}[!ht]
  \centering\includegraphics[width=\columnwidth]
  {images/search_comp_5000_pi.jpg}
  \caption{Search 5000 data 1000 times on Raspberry Pi
   (Scenario 11 compare with Scenario 29)}\label{f:fly}
\end{figure}


Compare insert 1000 data 1000 times on Google Cloud~\ref{f:fly}.

\begin{figure}[!ht]
  \centering\includegraphics[width=\columnwidth]
  {images/insert_comp_1000_google.jpg}
  \caption{Insert 1000 data 1000 times on Google Cloud
   (Scenario 13 compare with Scenario 31)}\label{f:fly}
\end{figure}

Compare insert 5000 data 1000 times on Google Cloud~\ref{f:fly}.

\begin{figure}[!ht]
  \centering\includegraphics[width=\columnwidth]
  {images/insert_comp_5000_google.jpg}
  \caption{Insert 5000 data 1000 times on Google Cloud
  (Scenario 14 compare with Scenario 32)}\label{f:fly}
\end{figure}

Compare insert 10000 data 1000 times on Google Cloud~\ref{f:fly}.

\begin{figure}[!ht]
  \centering\includegraphics[width=\columnwidth]
  {images/insert_comp_10000_google.jpg}
  \caption{Insert 10000 data 1000 times on Google Cloud
  (Scenario 15 compare with Scenario 33)}\label{f:fly}
\end{figure}

Compare search 1000 data 1000 times on Google Cloud~\ref{f:fly}.

\begin{figure}[!ht]
  \centering\includegraphics[width=\columnwidth]
  {images/search_comp_1000_google.jpg}
  \caption{Search 1000 data 1000 times on Google Cloud
  (Scenario 16 compare with Scenario 34)}\label{f:fly}
\end{figure}

Compare search 5000 data 1000 times on Google Cloud~\ref{f:fly}.

\begin{figure}[!ht]
  \centering\includegraphics[width=\columnwidth]
  {images/search_comp_5000_google.jpg}
  \caption{Search 5000 data 1000 times on Google Cloud
  (Scenario 17 compare with Scenario 34)}\label{f:fly}
\end{figure}

Compare search 10000 data 1000 times on Google Cloud~\ref{f:fly}.

\begin{figure}[!ht]
  \centering\includegraphics[width=\columnwidth]
  {images/search_comp_10000_google.jpg}
  \caption{Search 10000 data 1000 times on Google Cloud
  (Scenario 18 compare with Scenario 35)}\label{f:fly}
\end{figure}

Compare insert 1000 data 1000 times on Mac, Raspberry Pi and Google Cloud~\ref{f:fly}.

\begin{figure}[!ht]
  \centering\includegraphics[width=\columnwidth]
  {images/insert_comp_1000_three.jpg}
  \caption{Insert 1000 data 1000 times on  Mac, Raspberry Pi and Google Cloud
  (Scenario 1, 7 compare with Scenario 13)}\label{f:fly}
\end{figure}


Compare insert 5000 data 1000 times on Mac, Raspberry Pi and Google Cloud~\ref{f:fly}.

\begin{figure}[!ht]
  \centering\includegraphics[width=\columnwidth]
  {images/insert_comp_5000_three.jpg}
  \caption{Insert 5000 data 1000 times on  Mac, Raspberry Pi and Google Cloud
  (Scenario 2, 8 compare with Scenario 14)}\label{f:fly}
\end{figure}


Compare insert 10000 data 1000 times on Mac, Raspberry Pi and Google Cloud~\ref{f:fly}.

\begin{figure}[!ht]
  \centering\includegraphics[width=\columnwidth]
  {images/insert_comp_10000_three.jpg}
  \caption{Insert 10000 data 1000 times on  Mac, Raspberry Pi and Google Cloud
  (Scenario 3, 9 compare with Scenario 15)}\label{f:fly}
\end{figure}


\section{Conclusion}
Compare the performance of MongoDB and Cassandra on Macintosh. 

\subsection{Scenario 1, 4, 19 and 22}
Insert and search 1000 data 1000 times on Macintosh 
(Scenario 1,4 compare with Scenario 19,22):

The average insert time of mongoDB is 0.2 ms, while the average insert time of 
Cassandra is 0.4 ms.

The average search time of mongoDB is 0.007 ms, while the average insert time of 
Cassandra is 0.8 ms.

\subsection{Scenario 2, 5, 20 and 23}
Insert and search 5000 data 1000 times on Macintosh 
(Scenario 2, 5 compare with Scenario 20, 23):

The average insert time of mongoDB is 1.2 ms, while the average insert time of 
Cassandra is 0.4 ms.

The average search time of mongoDB is 0.03 ms, while the average insert time of 
Cassandra is 5 ms.

However, since this test are run on Macintosh parallely, they impact each other's 
performance.

\subsection{Scenario 3, 6, 21 and 24}
Insert and search 10000 data 1000 times on Macintosh 
(Scenario 3, 6 compare with Scenario 21, 24):

The average insert time of mongoDB is 2.2 ms, while the average insert time of 
Cassandra is 5 ms.

The average search time of mongoDB is 0.07 ms, while the average insert time of 
Cassandra is 10 ms.

Overall, Cassandra takes more time to insert and search, especially search. 

However, on Raspberry Pi the difference of their performance become closer.

The performance of MongoDB and Cassandra on Google Cloud is more stable.



\subsection{Scenario 7,10 ,25 and 28}
Insert and search 1000 data 1000 times on Raspberry Pi 
(Scenario 7, 10 compare with Scenario 25, 28):

The average insert time of mongoDB is 0.4 ms, while the average insert time of 
Cassandra is 0.5 ms.

The average search time of mongoDB is 0.007 ms, while the average insert time of 
Cassandra is 1.3 ms.

As we can see from the plot, the result is much more stable than w

\subsection{Scenario 8, 11, 26 and 29}
Insert and search 5000 data 1000 times on Raspberry Pi 
(Scenario 8,11 compare with Scenario 26, 29):

The average insert time of mongoDB is 2.0 ms, while the average insert time of 
Cassandra is 4.5 ms.

The average search time of mongoDB is 0.04 ms, while the average insert time of 
Cassandra is 6 ms.

However, since this test are run on Macintosh parallely, they impact each other's 
performance.

\subsection{Scenario 9, 12, 27 and 30}
Insert and search 10000 data 1000 times on Macintosh 
(Scenario 9,12 compare with Scenario 27, 30):

The average insert time of mongoDB is 4.5 ms, while the average insert time of 
Cassandra is 6 ms.

The average search time of mongoDB is 0.08 ms, while the average insert time of 
Cassandra is 14 ms.

Overall, Cassandra takes more time to insert and search, especially search. 

However, on Raspberry Pi performance of  both the MongoDB and Cassandra become worse.



\subsection{Scenario 13, 16, 31 and 34}
Insert and search 1000 data 1000 times on Google Cloud
(Scenario 13,16 compare with Scenario 31, 34):

The average insert time of mongoDB is 0.25 ms, while the average insert time of 
Cassandra is 0.5 ms.

The average search time of mongoDB is 0.006 ms, while the average insert time of 
Cassandra is 1.3 ms.

As we can see from the plot, the result is much more stable than running on Macintosh, 
because they are run serially. And no any other applications run at the same time. 

\subsection{Scenario 14, 17, 32 and 35}
Insert and search 5000 data 1000 times on Google Cloud 
(Scenario 14, 17 compare with Scenario 32, 35):

The average insert time of mongoDB is 1.2 ms, while the average insert time of 
Cassandra is 2.0 ms.

The average search time of mongoDB is 0.03 ms, while the average insert time of 
Cassandra is 6 ms.

With the amount of queries we insert and search every time,
the result is even more stable. 

\subsection{Scenario 15,18, 33 and 36}
Insert and search 10000 data 1000 times on Google Cloud 
(Scenario 15,18 compare with Scenario 33, 36):

The average insert time of mongoDB is 2.3 ms, while the average insert time of 
Cassandra is 5 ms.

The average search time of mongoDB is 0.07 ms, while the average insert time of 
Cassandra is 13 ms.

Overall, Cassandra takes more time to insert and search, especially search. 
And the performance of MongoDB and Cassandra on Google Cloud is more stable.


\subsection{Scenario 1, 7 and 13}
Insert 1000 data 1000 times on Mac, Raspberry Pi and Google Cloud
(Scenario 1, 7  compare with Scenario 13):

The average insert time on Mac  is 0.22 ms, the average insert time on Google Cloud
 is 0.25 ms while the average insert time on  Raspberry Pi is 0.4ms.

\subsection{Scenario 2, 8  and 14}
Insert 50000 data 1000 times on Mac, Raspberry Pi and Google Cloud
(Scenario 2, 8 compare with Scenario 14):

The average insert time on Mac is 1.22 ms, the average insert time on Google Cloud
 is 1.25 ms while the average insert time on  Raspberry Pi is 2.4ms.


\subsection{Scenario 3, 9 and 15}
Insert 10000 data 1000 times on Mac, Raspberry Pi and Google Cloud
(Scenario 3, 9 compare with Scenario 15):

The average insert time on Mac is 2.22 ms, the average insert time on Google Cloud
 is 2.4 ms while the average insert time on  Raspberry Pi is 5.4ms.
 
Overall,  performance on Raspberry Pi is  always the worst.

\begin{acks}

  The authors would like to thank Dr.~Gregor~von~Laszewski for his
  support and suggestions to write this paper.

\end{acks}

\bibliographystyle{ACM-Reference-Format}
\bibliography{report} 

